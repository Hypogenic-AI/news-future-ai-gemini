\section{Discussion}
\label{sec:discussion}

\paragraph{The Power of Grounding}
Our key finding is that LLM hallucination in creative writing can be controlled by rigorous grounding in formal criteria.
By treating the market resolution rules as a logical constraint satisfaction problem (solved by the \analyst) rather than a creative prompt, we force the creative generation (by the \journalist) to stay within the bounds of the predicted future.
This ``Constraint-then-Generate'' pattern is applicable beyond journalism to any domain requiring counterfactual reasoning.

\paragraph{Limitations}
Our system currently relies on the text description of markets.
Some markets rely on external links or implied context that the \analyst might miss.
Furthermore, as noted in the results, the system struggles to write ``serious'' news about ``silly'' markets.
Future work should incorporate a ``Tone Match'' classifier to align the article's seriousness with the market's intrinsic gravity.

\paragraph{Broader Implications}
\ours represents a step towards ``experiential forecasting.''
Instead of decision-makers staring at a dashboard of percentages, they could read the ``Morning Newspaper of 2026'' to intuitively grasp the consequences of current trends.
This has potential applications in policy making, risk assessment, and strategic planning.
